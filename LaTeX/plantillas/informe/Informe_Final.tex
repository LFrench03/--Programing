\documentclass{article}
\usepackage[utf8]{inputenc}
\usepackage{hyperref}
\usepackage{graphicx}
\usepackage{listings}
\usepackage{graphicx}
\usepackage[a4paper, left = 3cm, right = 3cm, top =2cm]{geometry}
\definecolor{codegreen}{rgb}{0,0.6,0}
\definecolor{codegray}{rgb}{0.5,0.5,0.5}
\definecolor{codepurple}{rgb}{0.58,0,0.82}
\definecolor{backcolour}{rgb}{0.95,0.95,0.92}
\definecolor{crimson}{HTML}{DC1420}
\definecolor{customGreen}{HTML}{228B22}
\definecolor{marron}{HTML}{804000}
\definecolor{orange}{HTML}{FFA500}
\lstdefinestyle{mystyle}{
    backgroundcolor=\color{backcolour},   
    commentstyle=\color{codegreen},
    keywordstyle=\color{magenta},
    numberstyle=\tiny\color{codegray},
    stringstyle=\color{codepurple},
    basicstyle=\ttfamily\footnotesize,
    breakatwhitespace=false,         
    breaklines=true,                 
    captionpos=b,                    
    keepspaces=true,                 
    numbers=left,                    
    numbersep=5pt,                  
    showspaces=false,                
    showstringspaces=false,
    showtabs=false,                  
    tabsize=2
}
\lstset{keepspaces=true, style=mystyle}

\title{\textcolor{crimson}{\Large{\textbf{Proyecto Final de Análisis Exploratorio de Datos}}} \\ \normalsize{\textit{Facultad de Matemática y Computación}} \\\normalsize{\textit{Ciencia de Datos \\ Grupo D111}}\\ \textcolor{customGreen}{\large{\textit{Dataset: Trees.}}}} 
\author{\textcolor{marron}{\normalsize{Guillermo Cepero García}} \\ \textcolor{marron}{\normalsize{Luis Ernesto Serras Rimada}} \\ \textcolor{marron}{\normalsize{Miguel Vadim Vilariño Pedraza}}}
\date{\today}
\begin{document}
\begin{titlepage}
    \begin{center}
    {\includegraphics[width=0.15\textwidth]{img/matcom.jpg}\par} %Logo matcom con presentacion del informe
    \vspace{0.1cm}
    {\bfseries\LARGE Ciencia de Datos \par}
    \vspace{0.2cm}
    {\scshape\Large  Facultad de Matamática y Computación\par}
    \vspace{0.6cm}
    {\scshape\Huge{\textbf{Asignatura \\ Proyecto Final}} \par}
    \vspace{0.2cm}
    {\itshape\Large {\large{\textit{Dataset: Trees.}} \par}
    \vspace{0.8cm}
    {\large {Integrantes:} \\ \normalsize{Nombre 1} \\ \normalsize{Nombre 2} \\ \normalsize{Nombre 3} \par}
    \vspace{0.5cm}
    {\includegraphics[width=0.85\textwidth]{img/imagen del tale.jpeg}}
    \end{center}
\end{titlepage}

\newpage		%Con \newpage pasas de pagina
\tableofcontents
\lstlistoflistings
\newpage
\maketitle
    Informe Plantilla
    \tableofcontents 	%enlaces de secciones, subsecciones y de codigos
    \lstlistoflistings	
    Introduccion
\newpage
\section{Desarrollo}	%secciones
Desarrollo\\
\begin{center} 	%centrado
\includegraphics[height = 15cm]{img/imagen.jpeg} %insertar imagenes
\end{center}
\newpage
\subsection{Code}
\begin{lstlisting}[language=Python, caption=Python example]	%Codigo
    import numpy as np
        
    def incmatrix(genl1,genl2):
        m = len(genl1)
        n = len(genl2)
        M = None #to become the incidence matrix
        VT = np.zeros((n*m,1), int)  #dummy variable
        
        #compute the bitwise xor matrix
        M1 = bitxormatrix(genl1)
        M2 = np.triu(bitxormatrix(genl2),1) 
    
        for i in range(m-1):
            for j in range(i+1, m):
                [r,c] = np.where(M2 == M1[i,j])
                for k in range(len(r)):
                    VT[(i)*n + r[k]] = 1;
                    VT[(i)*n + c[k]] = 1;
                    VT[(j)*n + r[k]] = 1;
                    VT[(j)*n + c[k]] = 1;
                    
                    if M is None:
                        M = np.copy(VT)
                    else:
                        M = np.concatenate((M, VT), 1)
                    
                    VT = np.zeros((n*m,1), int)
        
        return M
    \end{lstlisting}
\clearpage	%se usa esto para pasar a la penultima pagina

\section{Conclusiones}
\subsection{A}	%subsecciones

\cleardoublepage	%se usa esto para la pagina final

\section{Referencias}	%referencias
\url{https://github.com/LFrench03/Modelo-de-Crecimiento-Poblacional/blob/main/data/csv/poblacion-residente.csv}	%enlace con \url
\\\textit{\textbf{\underline{Referencias}}:}

\begin{itemize}	%listas
    \item (1)
\end{itemize}

\end{document}
